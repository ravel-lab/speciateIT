\documentclass{article}

\input{/Users/pgajer/organizer/publishing/latex/preamble.tex}

%% reference list of some usefull latex commands
%% /Users/pgajer/organizer/publishing/latex/templates.txt

% -------------------------- local variables -------------------------------------
% setting TEXINPUTS in .bash_profile suppose to be better than the next command; is it???

\newcommand\one{1}

\newcommand{\Li}{\emph{Lactobacillus iners}\xspace}
\newcommand{\Lc}{\emph{Lactobacillus crispatus}\xspace}
\newcommand{\Lg}{\emph{Lactobacillus genseri}\xspace}
\newcommand{\Lj}{\emph{Lactobacillus jensenii}\xspace}
\newcommand{\Gv}{\emph{Gardnerella vaginalis}\xspace}
\newcommand{\Av}{\emph{Atopobium_vaginae}\xspace}

\newcommand{\vbc}{vaginal bacterial community\xspace}
\newcommand{\vbcs}{vaginal bacterial communities\xspace}

\newcommand{\alr}{abundance log ratio\xspace}
\newcommand{\alrs}{abundance log ratios\xspace}
\newcommand{\mb}{metabolite\xspace}
\newcommand{\mbs}{metabolites\xspace}

\newcommand{\NNs}{nearest neighbors\xspace}

\newcommand{\MG}{metagenomic\xspace}
\newcommand{\XMG}{\ensuremath{X^{3}_{\mathrm{MG}}}\xspace}

\newcommand{\XMGZB}{\ensuremath{X^{3}_{\mathrm{MG}}(\mathrm{ZB})}\xspace}

\newcommand{\MT}{metatranscriptomics\xspace}
\newcommand{\MB}{metabolomics\xspace}

\newcommand{\mgBig}{\ensuremath{X^{1899}_{MG}}\xspace}
\newcommand{\mgSmall}{\ensuremath{X^{3}_{\mathrm{MG}}}\xspace}

\newcommand{\psPTB}{\ensuremath{p(\mathrm{sPTB})}\xspace}
\newcommand{\EpsPTB}{\ensuremath{Ep(\mathrm{sPTB})}\xspace}

\newcommand{\fHL}{\ensuremath{f_{\mathrm{HL}}}\xspace}
\newcommand{\fH}{\ensuremath{f_{\mathrm{H}}}\xspace}
\newcommand{\fL}{\ensuremath{f_{\mathrm{L}}}\xspace}

\newcommand{\seq}[2]{\ensuremath{#1_1, \ldots, #1_{#2}}\xspace}

\newcommand{\hHL}[2]{\ensuremath{\hat{H}_{#1}\hat{L}_{#2}}\xspace}

\newcommand{\cs}{community state\xspace}
\newcommand{\css}{community states\xspace}
\newcommand{\logT}{\ensuremath{\log_{10}}\xspace}
\newcommand{\kNN}{\ensuremath{\mathrm{kNN}}\xspace}
\newcommand{\logD}{\ensuremath{\log_{2}}\xspace}
\newcommand{\logRat}[2]{\ensuremath{\log_{10}(a(\mathrm{#1})/a(\mathrm{#2}))}\xspace}
\newcommand{\med}[2]{\ensuremath{m_{\mathrm{#1}}(\mathrm{#2})}\xspace}

\newcommand{\buildModelTree}{\verb+buildModelTree+}

\newcommand{\<}{\ensuremath{<}}
\renewcommand{\>}{\ensuremath{>}}

\newcommand{\picsDir}{/Users/pgajer/devel/speciateIT/pics}

%% -------------------------------------------------------------------------------

\title{speciateIT}
% \author{}
% \date{\vspace{0.1cm}\today}
\date{\today}

\begin{document}

\maketitle

speciateIT is a super fast 16S rRNA gene fragment species level classifier. For
each 16S rRNA amplicon region a reference set of high quality sequences is used
to generate Markov chain models for each taxonomic rank starting from specie
level, through genus, etc, up to the phylum level. A query sequence is first
classified at the phylum level. At each level the most probable model is
accepted if the log odds of the probability of a sequence being generated by the
model is above certain threshold. Whether or not speciateIT is able to assign
classification at the next taxonomic level depends on the ratio between the
posterior probability that the sequence comes from the given level model and the
posterior probability that the sequence comes from the model of one of the children
of the given taxonomic level. Due to high speed of speciateIT, the
classification can be done on all reads even in a large project.




\section*{buildModelTree}

\textbf{Input} The input to \verb+buildModelTree+ consists of
\tightlists
\begin{itemize}
\item A species taxon file with two columns: sequence ID and species name. The format of the
  file is shown below.
  \begin{verbatim}
  579041	Achromobacter_aegrifaciens
  1143416	Achromobacter_animicus
  241186	Achromobacter_animicus
  HE613448_1_1507	Achromobacter_animicus
  MF062573_1_1504	Achromobacter_animicus
  \end{verbatim}
\item A full taxon (or species lineage) tab delimited file a fragment of which
  is shown below.
  {\small
  \begin{verbatim}
BVAB1	g_Shuttleworthia	f_Lachnospiraceae	o_Clostridiales	c_Clostridia	p_Firmicutes	d_Bacteria
BVAB2	g_Acetivibrio	f_Ruminococcaceae	o_Clostridiales	c_Clostridia	p_Firmicutes	d_Bacteria
BVAB3	g_Acetivibrio	f_Ruminococcaceae	o_Clostridiales	c_Clostridia	p_Firmicutes	d_Bacteria
Dialister_sp._type_1	g_Dialister	f_Veillonellaceae	o_Clostridiales	c_Clostridia	p_Firmicutes	d_Bacteria
  \end{verbatim}}
  \item A reference fasta file of sequences representing species in the lineage
    file.
  \item The name of an output directory.
\end{itemize}
\vspace{7pt}
\textbf{Usage}
\begin{verbatim}
   buildModelTree -l <species lineage file> -i <fasta file> -t <species taxon file> -o <output dir>
\end{verbatim}

\vspace{5pt}
\noindent\textbf{Output} The program builds
\tightlists
\begin{itemize}
\item A model tree reflecting the species lineage data parent/child structure
  with leaf lables being species names and internal nodes corresponding to
  higher taxonomic ranks.
\item For each node of the model tree (except root) a fasta file of all ref seq's
   corresopnding to the node's subtree.
\item A file of absolute paths to just created fasta files. A file of taxonomic
   assignments of internal nodes
\end{itemize}

\section*{buildMC}

Given fasta files of training sequences, one or more non-negative integers, and
a fasta file of query sequences, \verb+buildMC+ builds Markov chain models for
sequences in each fasta file.

\textbf{Input}

\vspace{7pt}
\textbf{Usage: Building MC models on the fly}
\begin{verbatim}
buildMC-t <file of paths of tr'g fasta files> -k <k-mer size> -i <input fasta file> -o <output dir> [Options]
\end{verbatim}
Example
\begin{verbatim}
buildMC -t vaginal_319F_806R_nr_dir/spp_paths.txt -k 8 -i test.fa -o test_dir
\end{verbatim}

\textbf{Usage: Building MC models only}
\begin{verbatim}
buildMC-t <file of paths of tr'g fasta files> -k <k-mer size> -d < MC models dir> [Options]
\end{verbatim}
Examples
\begin{verbatim}
buildMC -t vaginal_319F_806R_nr_dir/spp_paths.txt -k 8 -d vaginal_319F_806R_nr_MCdir
buildMC -k 8 -d vaginal_319F_806R_nr_MCdir
\end{verbatim}

\textbf{Usage: Using prebuilt MC models}
\begin{verbatim}
buildMC-d <MC models directory> -i <input fasta file> -o <output dir> [Options]
\end{verbatim}
Example
\begin{verbatim}
buildMC -d vaginal_319F_806R_nr_MCdir -i test.fa -o test_dir
\end{verbatim}

\vspace{7pt}
\textbf{Options}
\tightlists
\begin{itemize}
\item -d \<dir\>     - directory for MC model files
\item -o \<dir\>     - output directory for MC taxonomy files
\item -i \<inFile\>  - input fasta file with sequences for which \verb+-log10(prob(seq | model_i))+ are to be computed
\item -t \<trgFile\> - file containing paths to training fasta files
\item -k \<K\>       - K is the k-mer size
\item --random-sample-size, -r \<n\> - number of random sequences to be generated
  for each MC model
\item --random-seq-length, -l \<n\>  - length of each random sequence
\item --pseudo-count-type, -p \<f\>
  \begin{itemize}
  \item f=0 for add 1 to all k-mer counts zero-offset
  \item f=1 for add $1/4^k$ to k-mer counts zero-offset
  \item f=2 the pseudocounts for a order k+1 model be alpha*probabilities from
    an order k model, recursively down to pseudo-counts of the letters for
    an order 0 model.
  \end{itemize}

\item -v - verbose mode
\item -h, --help      - this message
\end{itemize}

\vspace{5pt}
\noindent\textbf{Output} Conditional probabities tables \<file\_i\>.MC\<order\>.log10cProb

Output file format:
\begin{verbatim}
	seqId   model1        model2 ...
	seq_1   log10prob11   log10prob12  ...
	seq_2   log10prob21   log10prob22  ...
	...
\end{verbatim}
where log10prob\_ij is the \logT of  \verb+prob(seq_i | model_j)+.


\section*{est\_error\_thlds}

\verb+est_error_thlds+ estimating error thresholds.

\textbf{Usage}
\begin{verbatim}
est_error_thlds -d <MC models dir> [Options]
\end{verbatim}
\vspace{7pt}
\textbf{Options}
\tightlists
\begin{itemize}
\item -d \<dir\>   - directory containing MC model files and reference
  sequences fasta files.
\item --offset-coef \<x\>  - offset = \logT(x). Currently x=0.99.
\item --tx-size-thld \<x\> - currently 10.
\end{itemize}

Example
\begin{verbatim}
est_error_thlds -v -d Firmicutes_group_5_V3V4_MC_models_dir
\end{verbatim}

\noindent\textbf{Output} A file, error\_thlds.txt, in the MC models directory
with two columns \<taxonName\> and \<threshold value\>. An example of the output
file is shown below.
\begin{verbatim}
c_Actinobacteria	0.954243
o_Bacteroidales	0.954243
g_Chlamydia	0.954243
p_Firmicutes	0.954243
o_Fusobacteriales	0.954243
\end{verbatim}


\section*{classify}

Given a directory of MC model files, reference tree and a fasta file of query sequences,
\verb+classify+ classifies each sequence of the fasta file to a taxonomic rank corresponding to model
with the highest probability given that the $| \log( p(x | M_L) / p(x | M_R) | >
\theta$, where $\theta$ is a threshold.

\textbf{Usage}
\begin{verbatim}
  classify -d <MC models dir> -i <input fasta file> -o <output dir> [Options]
\end{verbatim}
Example
\begin{verbatim}
classify -d V3V4 -i my_V3V4_sample_sequences.fa -o out_V3V4
\end{verbatim}

 \vspace{7pt}
\textbf{Options}
\tightlists
\begin{itemize}
\item -d \<mcDir\>      - directory containing MC model files.
\item -o \<outDir\>     - output directory for MC taxonomy files.
\item -i \<inFile\> - input fasta file with sequences for which
  $-log10(prob(seq | model_i))$ are to be computed.
\item -r \<model tree\> - model tree with node labels corresponding to the names
  of the model files.
\item -t \<trgFile\>    - file containing paths to training fasta files.
\item -f \<fullTx\>     - fullTx file. Its optional parameter for printing
  classification output in a long format like in RDP classifier
\item -g \<faDir\>      - directory with reference fasta files.
\item --rev-comp, -c          - reverse complement query sequences before
  computing classification posterior probabilities.
\item --skip-err-thld         - classify all sequences to the species level.
\item --pp-embedding          - for each internal node report pp's of all
  children on the given sequence. Each internal node's table is written to a file \<node name\>\_ref\_lpps.txt (log posterior probabilities)
\item --max-num-amb-codes \<n\> - maximal acceptable number of ambiguity codes for
  a sequence above this number sequence's log10prob() is not computed and the
  sequence's id it appended to \<genus\>\_more\_than\_\<n\>\_amb\_codes\_reads.txt file.
  Default value: 5.
\item --pseudo-count-type, -p \<f\>
\begin{itemize}
\item  f=0 for add 1 to all k-mer counts zero-offset.
\item f=1 for add $1/4^k$ to k-mer counts zero-offset.
\item f=2 the pseudocounts for a order k+1 model be alpha*probabilities from an
  order k model, recursively down to pseudocounts of the letters for an order 0
  model.
\end{itemize}
\item -q, --quiet           - suppers pregress messages.
\item -v, --verbose         - verbose mode.
\item -h, --help            - this message.
\end{itemize}

\noindent\textbf{Output} A file  "MC\_order7\_results.txt"

The output file contains 4 columns: "Sequence ID", "Classification", "posterior
probability" and "number of Decisions".



\end{document}
%%% Local Variables:
%%% mode: latex
%%% TeX-master: t
%%% End:
